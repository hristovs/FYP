\chapter{Conclusions and Future Work}
To conclude, a summary is presented which describes the work done in the context of NDN and Blockchain. Finally, a Future Work section presents future projects which could result from this project.\\

\section{Conclusion}
In this body of work, Named Data Networking and Blockchain have been presented as separate entities. The strengths and weaknesses of both have been discussed in depth as well as their different components. \\
This paper concerns itself with certificate management. A certificate is every entity that produces Data.\\
This project presents a solution to NDN's problem of tasking each Face to verify a hierarchy of certificates. Currently in NDN, each Face, in order to verify the certificate they are interested in, must verify every certificate before it in the tree hierarchy structure of certificates. When it gets to the root certificate, the Face must send an Interest packet to the CA in order to verify the root cert. The problem with this implementation is the look-up time introduced by having to verify the root certificate. This \textbf{look-up} introduces considerable delays in a network.\\
This isn't always the case however, as NDN allows the network administrators to set up and configure the network's security in a number of different ways. They can be configured in a completely ad-hoc manner, where the Certificate Authority isn't a trusted authority, i.e. using an SDSI policy - Simple Distributed Security Infrastructure. This is mainly done in simulations in an experimental environment, and is neither secure or feasible to do in a real NDN deployment.\\

Alternatively, and more commonly in a real NDN deployment, networks use trusted Certificate Authorities such as Verisign, which grant certificates for local Certificate Authorities(root certs) which are then used to issue certificates to any entity that exists below the root in the namespace hierarchy of NDN. This is where the problem described above occurs, where a lot of needless communication between entities and the third-party Certificate Authority occurs \\

The solution presented in "Integration of Blockchain and Named-Data Networking" involves separate,dedicated nodes called miners, doing all of this verification on behalf of the network, appending the verified certificates to a Blockchain, and publishing and multicasting the verified chains to all of the nodes in a network which then no longer need to verify a hierarchy of certificates. This is because the Blockchain, by design, is tamper-proof, because all hash blocks in a Blockchain have been hashed with the previous block's hash and in order to tamper with or change a single block, a node would have to do all of the "Proof-of-Work" for all of the previously hashed blocks, which increases in difficulty exponentially. This is virtually impossible unless one has more computing power than an entire network of miners. This moves the single point of failure to all "miner" nodes. The network must guarantee that the miners' certificate verifications are correct(a trivial problem), which if done, would mean that no malicious node can tamper with or alter any certificate in the network.
\section{Future Work}
There are a number of directions this project could take. Firstly, it is important to note that the Blockchain doesn't hash blocks using a nonce to produce a set number of 0s. This means that the Blockchain algorithm doesn't do the "Proof-of-Work" described by Satoshi Nakamoto, required to create blocks in a regular Blockchain implementation. This is because this project has been a proof of concept solution. In future projects, this could be improved upon greatly, and the difficulty of the mining could be investigated to determine the trade-off of safety versus compute time. Since Blockchain isn't the main mechanism in making the network work, the way it is in a decentralized transaction system like Bitcoin, the hashing algorithm doesn't have to mimic the same difficulty. Additionally, a project could investigate the advantages and disadvantages of publishing the blockchain at a set time interval and which time intervals would best suit which networking topologies in ICN. In a Bitcoin environment, Blockchains are published once every 10 minutes. This particular time interval suits the dynamics of Bitcoin. However, how this affects an Information-Centric Network would be curious indeed.\par 
An important investigation would be to determine the best way to disseminate Blocks across the network. There are a couple of methods investigated and outlined in this project which can be found in chapter 3. \par 
Continuing on with the Blockchain data types - it's important to talk about C++ convention. The C++ language is constantly evolving and new revisions are released annually. Since C++11, smart pointers have become the de facto standard in memory allocation. It really is unacceptable to have a system where instances of a class or data type are created using the 'new' keyword and deleted using 'delete'. A future project would ameliorate the existing PibBlock data type and make it use either unique or shared pointers which rely on the C++ compiler to delete the allocated memory resources once the object is out of scope. \par 
Additionally, it could be beneficial to investigate making this solution into a piggy-back module for any network. \par
Finally, the main reason for this project is to investigate whether it is worth implementing a Blockchain in the Public Interest Base. It would be advisable to utilize a huge computing resource like the MacNeill or Stoker in order to generate realistic topologies on which then to test the Blockchain integration and whether it is worth implementing as a viable option for speeding up communications and reducing the load on the network.
