%%%%%%%%%%%%%%%%%%%%%%%%%%%%%%%%%%%%%%%%%%%%%%%%%%%%%%%%%%%%%%%%%%%%%%%%%%%%%
%%%
%%% File: utthesis2.doc, version 2.0jab, February 2002
%%%
%%% Based on: utthesis.doc, version 2.0, January 1995
%%% =============================================
%%% Copyright (c) 1995 by Dinesh Das.  All rights reserved.
%%% This file is free and can be modified or distributed as long as
%%% you meet the following conditions:
%%%
%%% (1) This copyright notice is kept intact on all modified copies.
%%% (2) If you modify this file, you MUST NOT use the original file name.
%%%
%%% This file contains a template that can be used with the package
%%% utthesis.sty and LaTeX2e to produce a thesis that meets the requirements
%%% of the Graduate School of The University of Texas at Austin.
%%%
%%% All of the commands defined by utthesis.sty have default values (see
%%% the file utthesis.sty for these values).  Thus, theoretically, you
%%% don't need to define values for any of them; you can run this file
%%% through LaTeX2e and produce an acceptable thesis, without any text.
%%% However, you probably want to set at least some of the macros (like
%%% \thesisauthor).  In that case, replace "..." with appropriate values,
%%% and uncomment the line (by removing the leading %'s).
%%%
%%%%%%%%%%%%%%%%%%%%%%%%%%%%%%%%%%%%%%%%%%%%%%%%%%%%%%%%%%%%%%%%%%%%%%%%%%%%%

\documentclass[a4paper, 12pt, oneside]{report}         %% LaTeX2e document.
\usepackage {tcdthesis}              %% Preamble.
\usepackage{graphicx,color}
\usepackage{anysize}
\usepackage{nth}
\usepackage[table]{xcolor}
\usepackage{array,ragged2e}
\usepackage{amsmath}
\usepackage{longtable}
\mastersthesis                     %% Uncomment one of these; if you don't
%\phdthesis                         %% use either, the default is \phdthesis.

\thesisdraft                       %% Uncomment this if you want a draft
                                     %% version; this will print a timestamp
                                     %% on each page of your thesis.

\leftchapter                       %% Uncomment one of these if you want
%\centerchapter                      %% left-justified, centered or
% \rightchapter                      %% right-justified chapter headings.
                                     %% Chapter headings includes the
                                     %% Contents, Acknowledgments, Lists
                                     %% of Tables and Figures and the Vita.
                                     %% The default is \centerchapter.

% \singlespace                       %% Uncomment one of these if you want
% \oneandhalfspace                   %% single-spacing, space-and-a-half
 \doublespace                       %% or double-spacing; the default is
                                     %% \oneandhalfspace, which is the
                                     %% minimum spacing accepted by the
                                     %% Graduate School.

\renewcommand{\thesisauthor}{Samuil Hristov}            %% Your official UT name.
\renewcommand{\thesismonth}{April}                  %% Your month of graduation.
\renewcommand{\thesisyear}{2019}                      %% Your year of graduation.
\renewcommand{\thesistitle}{\tiny{} \\ \LARGE{Integration of Blockchain and Named-Data Networking}}            %% The title of your thesis; use mixed-case.
\renewcommand{\thesisauthorpreviousdegrees}{ }  %% Your previous degrees, abbreviated; separate multiple degrees by commas.
\renewcommand{\thesissupervisor}{Dr. Stefan Weber}      %% Your thesis supervisor; use mixed-case and don't use any titles or degrees.
% \renewcommand{\thesiscosupervisor}{}                %% Your PhD. thesis co-supervisor; if any.

% \renewcommand{\thesiscommitteemembera}{}
% \renewcommand{\thesiscommitteememberb}{}
% \renewcommand{\thesiscommitteememberc}{}
% \renewcommand{\thesiscommitteememberd}{}
% \renewcommand{\thesiscommitteemembere}{}
% \renewcommand{\thesiscommitteememberf}{}
% \renewcommand{\thesiscommitteememberg}{}
% \renewcommand{\thesiscommitteememberh}{}
% \renewcommand{\thesiscommitteememberi}{}

\renewcommand{\thesisauthoraddress}{Dublin, Ireland}

%\renewcommand{\thesisdedication}{...}     %% Your dedication, if you have one; use "\\" for linebreaks.


%%%%%%%%%%%%%%%%%%%%%%%%%%%%%%%%%%%%%%%%%%%%%%%%%%%%%%%%%%%%%%%%%%%%%%%%%%%%%
%%%
%%% The following commands are all optional, but useful if your requirements
%%% are different from the default values in utthesis.sty.  To use them,
%%% simply uncomment (remove the leading %) the line(s).

% \renewcommand{\thesiscommitteesize}{...}
                                     %% Uncomment this only if your thesis
                                     %% committee does NOT have 5 members
                                     %% for \phdthesis or 2 for \mastersthesis.
                                     %% Replace the "..." with the correct
                                     %% number of members.

\renewcommand{\thesisdegree}{Bachelor of Arts(Mod.)}  %% Uncomment this only if your thesis
                                     %% degree is NOT "DOCTOR OF PHILOSOPHY"
                                     %% for \phdthesis or "MASTER OF ARTS"
                                     %% for \mastersthesis.  Provide the
                                     %% correct FULL OFFICIAL name of
                                     %% the degree.

\renewcommand{\thesisdegreeabbreviation}{B.A.(Mod.)}
                                     %% Use this if you also use the above
                                     %% command; provide the OFFICIAL
                                     %% abbreviation of your thesis degree.

\renewcommand{\thesistype}{Dissertation}    %% Use this ONLY if your thesis type
                                     %% is NOT "Dissertation" for \phdthesis
                                     %% or "Thesis" for \mastersthesis.
                                     %% Provide the OFFICIAL type of the
                                     %% thesis; use mixed-case.

% \renewcommand{\thesistypist}{...}  %% Use this to specify the name of
                                     %% the thesis typist if it is anything
                                     %% other than "the author".

%%%
%%%%%%%%%%%%%%%%%%%%%%%%%%%%%%%%%%%%%%%%%%%%%%%%%%%%%%%%%%%%%%%%%%%%%%%%%%%%%



\begin{document}                                  %% BEGIN THE DOCUMENT
\begin{titlepage}
\begin{center}
\huge{University of Dublin}\\[2.5mm]
\includegraphics[width=1.2in]{trinity.jpg}\\
\Huge{TRINITY COLLEGE} \\
\vspace{1in}
\Large\textbf{Integration of Blockchain and Named-Data Networking}\\
\vspace{0.5in}
Samuil Hristov\\
Final Year Project April 2019\\
Supervisor: Dr. Stefan Weber\\
\vfill
School of Computer Science and Statistics\\
O'Reilly Institute, Trinity College, Dublin 2, Ireland\\
\end{center}
\end{titlepage}
\thesisdeclarationpage				  %% Generate the declaration page.

\thesispermissionpage				  %% Generate the copyright permission page

%\thesisdedicationpage                             %% Generate the dedication page.

\begin{thesisacknowledgments}                     %% Use this to write your
 todo
\end{thesisacknowledgments}                       %% allowed in LaTeX2e par-mode.

\begin{thesisabstract}

\textbf{Abstract:} Information-Centric Networking (ICN) is a communication approach that 
makes content 'living' in a network the focus of communication, in 
contrast to the traditional communication between hosts based on IP 
addresses. In order to ensure the validity of content, it should be 
signed by its producer and  in order to ensure that information is only 
accessible to a select number of consumers, content may need to be 
encrypted. Named-Data Networking (NDN) is an ICN implementation that 
provides a framework for the exchange of named content and a 
certificate-based security mechanism to sign and encrypt/decrypt 
content. The certificates in NDN are held at individual nodes and have 
to be requested by other nodes in a network in order to verify content, 
leading to additional latency once content has been retrieved.

This project has extended NDN's certificate management system by 
distributing certificates based on a distributed ledger i.e. a 
blockchain. Transactions such as the creation and removal are announced 
by nodes to miners which incorporate the transaction into new blocks and 
distribute these to nodes for inclusion into the Blockchain. Nodes have 
access to the current set of certificates through their Blockchain and 
can verify content through these certificates.

NDN currently implements a signature verification standard by defining a common certificate format. All signed data transfers in NDN are certificates because they contain a public key.X509 Format.
\end{thesisabstract}

\tableofcontents                                  %% Generate table of contents.
\listoftables                                     %% Uncomment this to generate list of tables.
\listoffigures                                    %% Uncomment this to generate list of figures.

%%
%% Include thesis chapters here...
%%
  \chapter{Introduction}
Named Data Networking is an interesting new paradigm in the space of network architectures. Years on since Bell Labs' work on telephony, networking research had assumed that telephony is the right model for data networking\cite{01}. This model implied that there had to be a determined route between two points for communication to happen - the setup for which was costly. This was proven not to be the case in Paul Baran's research paper titled ``On Distributed Communication Networks" in 1964, which was widely disregarded until the practical application of his work in ARPAnet(Sept,'71).MIT Senior Researcher David Clark's paper on end-to-end principle confirmed the same and it became apparent that networking solved the telephony problem. However, we are still using this same architecture which was invented as a solution for a problem that is now five decades old, and the Internet of today is not facing the same challenges. CISCO predicted in 2013\cite{02}\cite{03}, that by the end of 2017, the annual traffic of the internet would exceed 1.4 zettabytes with almost 80\% of that being video traffic.

This is why, in the age of content delivery, Named Data Networking and other Information Centric Networking architectures aim to move away from the source-destination pairwise method of IP communication which is inherently limited. Instead, NDN proposes a Name-based approach where each node in a network can request content based on the name of a piece of data it requires. 

This project aims to improve on the security implemented in Named Data Networking.


The aim of this introduction is to give background for motivation as well as lay out the structure of this paper.
\section{Motivation}
Traditional NDN: The traditional NDN architecture presents a security architecture not dissimilar to the Central Authority architecture conceptualized by Loren Kohnfelder in 1978\cite{04}. It provides a public file\cite{05} system where all nodes can check the entries for other nodes issued by a central, trusted third party called the Certificate Authority(CA) which signs each entry or `certificate'. This is all quite simple in a MiniNDN test environment, as described in the Certificates section in Chapter 2. In the case of a general or non-experimental NDN deployment\cite{06} however, this is not the case.\par 
The Network Manager would have to request a certificate verified by the CA. The manager, in turn, must also trust the CA and therefore the CA must be authentic and trusted by most networks. Trusted companies that provide this CA service include Verisign, GeoTrust,Symantec, etc... \par
The Network Manager would then provide a certificate for every node in the network which would in turn be signed by the root certificate provided by the authenticated CA. When a node wants to verify a certificate, like it would if it was receiving a Data packet, it would have to verify all of the certificates in the network hierarchy meaning it would also have to verify the root network certificate, for which it would have to contact the Certificate Authority. This means that verifying certificates can potentially involve a lot of communication.\par 
This could be avoided by introducing a Blockchain which each node could keep a copy of. This would allow nodes to verify certificates without needing to contact the CA.

To visualize this, the following example is presented in Figure 1.1. An NDN Network with an external CA has a root namespace \textbf{/ndn/} and in it there is a node called A. The root namespace is issued a certificate from the Certificate Authority which in this example is external e.g: Verisign. The root then issues a certificate for Node A. Node A in turn issues a certificate for the clock module in its namespace. These certificates allow all of the entities in the network to verify that their data is their own. If another node, like Node B for example, wants the time from Node A's clock, the node would simply express an Interest. When the clock replies with a Data packet, Node B would have to verify the data by verifying clock.cert, a.cert and root.cert. In order to verify root.cert,node B would need to contact the CA - which in the case of a real NDN environment might well be Verisign.\par
 This project recognizes that this process isn't efficient use of bandwidth as nodes would constantly have to contact the CA, performing look-ups for the sake of root certificate verification and that this can be mitigated if each node had a Blockchain of verified certificates at its disposal.
\begin{figure}[ht]
\centering
\includegraphics[width=6in,left]{certarch.png}
\caption{Certificate Hierarchy}
\end{figure}

\section{Aims}
Blockchain was an architecture proposed to store and verify transactions \textbf{reliably} in a decentralized currency system. Instead of transactions, this project aims to store certificates in a similar fashion. The goal is to do so efficiently, without increasing computational load on individual nodes in the system or increasing significantly the bandwidth use.

 It is important to note that there isn't a monetary incentive for doing this ``Proof-of-Work"\cite{07} so each network should have a dedicated group of miners which verify blocks. The assumption here is that all(or most) of the nodes will not be malicious and will not be pooling their resources to attack the network. This means that in order for one to alter the list of certificates, they would have to have more computing power than the entire network of miners. This will allow for safer communication between nodes in a network. 

\section{Road-map}
\chapquote{``Begin at the beginning," the King said gravely, ``and go on till you come to the end: then stop."}{Lewis Carroll}{Alice in Wonderland}

This paper is structured as follows: State of the Art(Lit. Review), Design and Implementation, and Evaluation.
\begin{itemize}
\item Chapter 2 contains the background information required for the scope of this project.It also contains literary reviews of the papers which discuss the State of the Art. It looks at a ranking system for the papers reviewed and also provides critique for each one. 

\item Chapter 3 discusses in detail the design of the Blockchain solution in NDN. It outlines all aspects of the conception of the data types, including solutions, design challenges and alterations that were made along the way.

\item Chapter 4 describes the implementation of this paper's solution

\item Chapter 5 goes on to evaluate the working solution by discussing different experiments and topologies. It presents graphs which illustrate the performance differences in different topologies

\end{itemize}
The paper concludes with Chapter 6 which sums up the work that's been presented and outlines any future work that might be undertaken regarding the project.                                
  \include{chapter2}                                
  \include{chapter3}
  \chapter{Results}
\section{Ideal Evaluation}
\subsection{Overhead and Latency}
\section{Discussion}


  \chapter{Conclusions and Future Work}
To conclude, a summary is presented which 
\section{Future Work}
There are a number of directions this project could take.

  \chapter{Conclusions and Future Work}
To conclude, a summary is presented which describes the work done in the context of NDN and Blockchain. Finally, a Future Work section presents future projects which could result from this project.\\

\section{Conclusion}
In this body of work, Named Data Networking and Blockchain have been presented as separate entities. The strengths and weaknesses of both have been discussed in depth as well as their different components. \\
This paper concerns itself with certificate management. A certificate is every entity that produces Data.\\
This project presents a solution to NDN's problem of tasking each Face to verify a hierarchy of certificates. Currently in NDN, each Face, in order to verify the certificate they are interested in, must verify every certificate before it in the tree hierarchy structure of certificates. When it gets to the root certificate, the Face must send an Interest packet to the CA in order to verify the root cert. The problem with this implementation is the look-up time introduced by having to verify the root certificate. This \textbf{look-up} introduces considerable delays in a network.\\
This isn't always the case however, as NDN allows the network administrators to set up and configure the network's security in a number of different ways. They can be configured in a completely ad-hoc manner, where the Certificate Authority isn't a trusted authority, i.e. using an SDSI policy - Simple Distributed Security Infrastructure. This is mainly done in simulations in an experimental environment, and is neither secure or feasible to do in a real NDN deployment.\\

Alternatively, and more commonly in a real NDN deployment, networks use trusted Certificate Authorities such as Verisign, which grant certificates for local Certificate Authorities(root certs) which are then used to issue certificates to any entity that exists below the root in the namespace hierarchy of NDN. This is where the problem described above occurs, where a lot of needless communication between entities and the third-party Certificate Authority occurs \\

The solution presented in "Integration of Blockchain and Named-Data Networking" involves separate,dedicated nodes called miners, doing all of this verification on behalf of the network, appending the verified certificates to a Blockchain, and publishing and multicasting the verified chains to all of the nodes in a network which then no longer need to verify a hierarchy of certificates. This is because the Blockchain, by design, is tamper-proof, because all hash blocks in a Blockchain have been hashed with the previous block's hash and in order to tamper with or change a single block, a node would have to do all of the "Proof-of-Work" for all of the previously hashed blocks, which increases in difficulty exponentially. This is virtually impossible unless one has more computing power than an entire network of miners. This moves the single point of failure to all "miner" nodes. The network must guarantee that the miners' certificate verifications are correct(a trivial problem), which if done, would mean that no malicious node can tamper with or alter any certificate in the network.
\section{Future Work}
There are a number of directions this project could take. Firstly, it is important to note that the Blockchain doesn't hash blocks using a nonce to produce a set number of 0s. This means that the Blockchain algorithm doesn't do the "Proof-of-Work" described by Satoshi Nakamoto, required to create blocks in a regular Blockchain implementation. This is because this project has been a proof of concept solution. In future projects, this could be improved upon greatly, and the difficulty of the mining could be investigated to determine the trade-off of safety versus compute time. Since Blockchain isn't the main mechanism in making the network work, the way it is in a decentralized transaction system like Bitcoin, the hashing algorithm doesn't have to mimic the same difficulty. Additionally, a project could investigate the advantages and disadvantages of publishing the blockchain at a set time interval and which time intervals would best suit which networking topologies in ICN. In a Bitcoin environment, Blockchains are published once every 10 minutes. This particular time interval suits the dynamics of Bitcoin. However, how this affects an Information-Centric Network would be curious indeed.\par 
An important investigation would be to determine the best way to disseminate Blocks across the network. There are a couple of methods investigated and outlined in this project which can be found in chapter 3. \par 
Continuing on with the Blockchain data types - it's important to talk about C++ convention. The C++ language is constantly evolving and new revisions are released annually. Since C++11, smart pointers have become the de facto standard in memory allocation. It really is unacceptable to have a system where instances of a class or data type are created using the 'new' keyword and deleted using 'delete'. A future project would ameliorate the existing PibBlock data type and make it use either unique or shared pointers which rely on the C++ compiler to delete the allocated memory resources once the object is out of scope. \par 
Additionally, it could be beneficial to investigate making this solution into a piggy-back module for any network. \par
Finally, the main reason for this project is to investigate whether it is worth implementing a Blockchain in the Public Interest Base. It would be advisable to utilize a huge computing resource like the MacNeill or Stoker in order to generate realistic topologies on which then to test the Blockchain integration and whether it is worth implementing as a viable option for speeding up communications and reducing the load on the network.

  \include{chapter7}


%\addcontentsline {toc}{chapter}{Appendices}       %% Force Appendices to appear in contents
\begin{appendix}
\chapter{Abbreviations}

\begin{longtable}{p{40mm}|p{100mm}}
	\textbf{Short Term}&\textbf{Expanded Term}\\
	\hline
	CA & Central Authority\\
	DNS & Domain Name System\\
	ACK & Packet Acknowledgement \\ 
	IoT & Internet of Things \\  
	URI & Universal Resource Indicator \\  
	ICN & Infomration Centric Networks\\ 
	NDN & Named Data Networking \\ 
	CCN & Content Centric Networking\\ 
	NDN-CXX & C++ Library with eXperimental eXtensions \\ 
	NFD & Named Data Forwarding Daemon \\ 
	NLSR & Named Data Link State Routing \\ 
	LSA & Link State Advertisement \\ 
	HR & Hyperbolic Routing \\  
	Face & Interface(Physical/Logical)\\ 
	FIB & Forward Interest Base \\ 
	CS & Content Store(Cache) \\ 
	PIT & Pending Interest Table \\ 
	PIB & Public Information Base \\ 
	RIB & Routing Information Base \\ 
	PKI & Public Key Infrastructure \\ 
	TPM & Trusted Platform Module \\ 
	TLV & Type Length Value Encoding \\ 
	TCP & Transmission Control Protocol \\ 
	IP & Internet Protocol  \\ 
	LNPM & Longest Name Prefix Matching \\
	DHCP & Dynamic Host Configuration Protocol \\ 
	ARP & Address Resolution Protocol \\ 
	DNS & Domain Name System \\ 
	MAC & Media Access Control \\ 
	RTT & Round Trip Time \\ 
	RIB & Routing Information Base\\
	SDSI & Simple Distributed Security Infrastructure
\end{longtable}

%\include{appendix2}
\end{appendix}


%\addcontentsline {toc}{chapter}{Bibliography}     %% Force Bibliography to appear in contents

\begin{thebibliography}{ieeetr}                   %% Start your bibliography here; you can
%\bibliography{refs}                               %% also use the \bibliography command



\bibitem{01}
[Zhang17a] ``An Overview of Named Data Networking by Lixia Zhang." Proceedings of
MILCOM 2017. Baltimore, MD, USA. Oct 2017.

\bibitem{02}
[Ioannou16] ``A Survey of Caching Policies and Forwarding Mechanisms in Information-Centric Networking" by Andrianna Ioannou \& Stefan Weber. IEEE Communications Surveys \& Tutorials. Volume:18, Issue:4, Q4 2016.

\bibitem{03}
[C.Index13] ``Cisco visual networking index: Forecast and methodology 2012-2017" by C. Index, White Paper. May 2013.


\bibitem{04}
[Kohnfelder78a] ``Towards a Practical Public-Key Cryptosystem" by Loren M. Kohnfelder. B.Sc. Dissertation. MIT, Boston, MA, USA, May 1978.

\bibitem{05}
[Kohnfelder78b] ``Towards a Practical Public-Key Cryptosystem" by Loren M. Kohnfelder. B.Sc. Dissertation. MIT, Boston, MA, USA, May 1978.

\bibitem{06}
[Weber19] ``Thesis first draft corrections" by Stefan Weber. TCD, Dublin, Apr 2019.

\bibitem{07}
[Nakamoto09a] ``Bitcoin: A Peer-to-Peer Electronic Cash System" by Satoshi Nakamoto, Oct 2008.  \texttt{\url{https://bitcoin.org/bitcoin.pdf}}

\bibitem{08}
[Nakamoto09b] ``Bitcoin: A Peer-to-Peer Electronic Cash System" by Satoshi Nakamoto, Oct 2008.  \texttt{\url{https://bitcoin.org/bitcoin.pdf}}


\bibitem{09}
[Nakamoto09c] ``Bitcoin: A Peer-to-Peer Electronic Cash System" by Satoshi Nakamoto, Oct 2008.  \texttt{url{https://bitcoin.org/bitcoin.pdf}}



\bibitem{010}
[Jacobson06a] ``A New Way to Look at Networking" by Van Jacobson. Google Tech Talks Archive, Aug 2006. \texttt{\url{https://youtu.be/oCZMoY3q2uM}}

\bibitem{011}
[Jacobson06b]  ``A New Way to Look at Networking" by Van Jacobson. Google Tech Talks Archive, Aug 2006. \texttt{\url{https://youtu.be/oCZMoY3q2uM}}


\bibitem{012}
[Medium] ``How The Byzantine General Sacked the Castle - A Look into Blockchain" by Debraj Ghosh. Medium.com, Apr 2016.


\bibitem{013}
[Shostack82] ``The Byzantine Generals Problem" by L. Lamport, R.Shostack \& M. Pease. ACM Transactions on Programming Languages and Systems, Vol. 4, No. 3. Jul 1982.


\bibitem{014}
[Nakamoto09d] ``Bitcoin: A Peer-to-Peer Electronic Cash System" by Satoshi Nakamoto, Oct 2008.  \texttt{\url{https://bitcoin.org/bitcoin.pdf}}


\bibitem{015}
[Nakamoto09e] ``Bitcoin: A Peer-to-Peer Electronic Cash System" by Satoshi Nakamoto, Oct 2008.  \texttt{\url{https://bitcoin.org/bitcoin.pdf}}


\bibitem{016}
[Zhang17b] ``An Overview of Named Data Networking by Lixia Zhang." Proceedings of
MILCOM 2017. Baltimore, MD, USA, Oct 2017. 

\bibitem{017}
[Zhang14] ``Named Data Networking" by Lixia Zhang, Van Jacobson \& Alexander Afanasyev. Proceedings of ACM SIGCOMM 2014. Chicago, IL, USA, Aug 2014.

\bibitem{018}
[Named-Data.net] ``Name Docs" by NDN Team. NDN Packet Format Specification. \texttt{\url{https://named-data.net/doc/NDN-packet-spec/current/name.html}}

\bibitem{019}
[Yuan15a] ``Reliably Scalable Name Prefix Lookup" by Haowei Yuan \& Patrick Crowley.Proceedings of ACM/IEEE Symposium on Architectures for Networking and Communication Systems(ANCS 2015). Oakland, CA, USA.	May 2015.

\bibitem{020}
[Wikipedia] ``Example for Longest Prefix Matching." \texttt{\url{https://en.wikipedia.org/wiki/Longest_prefix_match}}

\bibitem{021}
[Yuan15b] ``Reliably Scalable Name Prefix Lookup" by Haowei Yuan \& Patrick Crowley.Proceedings of ACM/IEEE Symposium on Architectures for Networking and Communication Systems(ANCS 2015). Oakland, CA, USA.	May 2015.

\bibitem{022}
[Yuan15c] ``Reliably Scalable Name Prefix Lookup" by Haowei Yuan \& Patrick Crowley.Proceedings of ACM/IEEE Symposium on Architectures for Networking and Communication Systems(ANCS 2015). Oakland, CA, USA.	May 2015.

\bibitem{023}
[Rainer18a] ``Challenges and Opportunities of Named Data Networking in Vehicle-To-Everything Communication: A Review" by Benjamin Rainer \& Stefan Petscharing. Center for Safety and Communications Technologies, Austrian Institute of Technology. Vienna, Austria. October 2018.


\bibitem{024}
[ndn-docs] ``Face Class" by NDN Team. NDN Common Client Libraries API 0.6.5. documentation. \texttt{\url{https://named-data.net/doc/ndn-ccl-api/face.html}}

\bibitem{025}
[V2E paperb] ``Challenges and Opportunities of Named Data Networking in Vehicle-To-Everything Communication: A Review" by Benjamin Rainer \& Stefan Petscharing. Center for Safety and Communications Technologies, Austrian Institute of Technology. Vienna, Austria. October 2018.



\bibitem{026}
[V2E paperc] ``Challenges and Opportunities of Named Data Networking in Vehicle-To-Everything Communication: A Review" by Benjamin Rainer \& Stefan Petscharing. Center for Safety and Communications Technologies, Austrian Institute of Technology. Vienna, Austria. October 2018.


\bibitem{027}
[Jacobson09a] ``Networking Named Content" by Van Jacobson, Diana Smetters, James Thornton, Michael Plass, Nicholas Briggs \& Rebecca Braynard. Proceedings of the \nth{5} International Conference on Emerging Networking Experiments and Technologies. Rome, Italy. December 2009.

\bibitem{028}
[Jacobson09b] ``Networking Named Content" by Van Jacobson, Diana Smetters, James Thornton, Michael Plass, Nicholas Briggs \& Rebecca Braynard. Proceedings of the \nth{5} International Conference on Emerging Networking Experiments and Technologies. Rome, Italy. December 2009.

\bibitem{029}
[NFDTeam16] ``NFD Developer's Guide" by the NFD Team. NDN Technical Report, NDN-0021. Oct 2016.

\bibitem{030}
[Wang18a] ``A Secure Link State Routing Protocol for NDN" by Lan Wang, Vince Lehman, A.K.M. Mahmudul Hoque, Beichuan Zhang, Yingdi Yu \& Lixia Zhang. In IEEE Access, Vol. 6. Jan 2018.

\bibitem{031}
[Wang18b] ``A Secure Link State Routing Protocol for NDN" by Lan Wang, Vince Lehman, A.K.M. Mahmudul Hoque, Beichuan Zhang, Yingdi Yu \& Lixia Zhang. In IEEE Access, Vol. 6. Jan 2018.


\bibitem{032}
[Wang18c] ``A Secure Link State Routing Protocol for NDN" by Lan Wang, Vince Lehman, A.K.M. Mahmudul Hoque, Beichuan Zhang, Yingdi Yu \& Lixia Zhang. In IEEE Access, Vol. 6. Jan 2018.
 


\bibitem{033}
[Wang18d] ``A Secure Link State Routing Protocol for NDN" by Lan Wang, Vince Lehman, A.K.M. Mahmudul Hoque, Beichuan Zhang, Yingdi Yu \& Lixia Zhang. In IEEE Access, Vol. 6. Jan 2018.
 


\bibitem{034}
[Wang18e] ``A Secure Link State Routing Protocol for NDN" by Lan Wang, Vince Lehman, A.K.M. Mahmudul Hoque, Beichuan Zhang, Yingdi Yu \& Lixia Zhang. In IEEE Access, Vol. 6. Jan 2018.
 

\bibitem{035}
[Wang18f] ``A Secure Link State Routing Protocol for NDN" by Lan Wang, Vince Lehman, A.K.M. Mahmudul Hoque, Beichuan Zhang, Yingdi Yu \& Lixia Zhang. In IEEE Access, Vol. 6. Jan 2018.
 

\bibitem{036}
[Afanasyev13a] ``Let's ChronoSync: Decentralized Dataset State
Synchronization in Named Data Networking" by Alexander Afanasyev \& Zhenkai Zhu.Proceedings of \nth{21} IEEE International Conference on Network Protocols(ICNP 2013).Göttingen, Germany. Oct 2013.


\bibitem{037}
[Afanasyev13b] ``Let's ChronoSync: Decentralized Dataset State
Synchronization in Named Data Networking" by Alexander Afanasyev \& Zhenkai Zhu.Proceedings of \nth{21} IEEE International Conference on Network Protocols(ICNP 2013).Göttingen, Germany. Oct 2013.


\bibitem{038}
[Ioannou14a] ``Towards On-Path Caching Alternatives in Information-Centric Networks" by Andrianna Ioannou \& Stefan Weber. Proceedings of 39th Annual IEEE Conference on Local Computer Networks. Edmonton, AB, Canada. Sept 2014.


\bibitem{039}
[Weber15] ``Towards On-Path Caching Alternatives in Information-Centric Networks" by Andrianna Ioannou \& Stefan Weber. Proceedings of 39th Annual IEEE Conference on Local Computer Networks. Edmonton, AB, Canada. Sept 2014.


\bibitem{040}
[Merkle79a] ``Protocols for Public-Key Cryptosystems" by Ralph Merkle. In 1980 IEEE Symposium on Security and Privacy. Oakland, CA, USA. Apr 1980.

\bibitem{041}
[Spyridon18a] ``An Overview of Security Support in Named Data Networking" by Spyridon Mastorakis, Zhiyi Zhang, Yingdi Yu, Haitao Zhang, Eric Newberry, Yanbiao Li, Alexander Afanasyev \& Lixia Zhang. NDN Technical Report, NDN-0057, Rev. 2. Apr. 2018.
\bibitem{042}
[Spyridon18b] ``An Overview of Security Support in Named Data Networking" by Spyridon Mastorakis, Zhiyi Zhang, Yingdi Yu, Haitao Zhang, Eric Newberry, Yanbiao Li, Alexander Afanasyev \& Lixia Zhang. NDN Technical Report, NDN-0057, Rev. 2. Apr. 2018.

\bibitem{043}
[Spyridon18c] ``An Overview of Security Support in Named Data Networking" by Spyridon Mastorakis, Zhiyi Zhang, Yingdi Yu, Haitao Zhang, Eric Newberry, Yanbiao Li, Alexander Afanasyev \& Lixia Zhang. NDN Technical Report, NDN-0057, Rev. 2. Apr. 2018.

\bibitem{044}
[Spyridon18d] ``An Overview of Security Support in Named Data Networking" by Spyridon Mastorakis, Zhiyi Zhang, Yingdi Yu, Haitao Zhang, Eric Newberry, Yanbiao Li, Alexander Afanasyev \& Lixia Zhang. NDN Technical Report, NDN-0057, Rev. 2. Apr. 2018.

\bibitem{045}
[Spyridon18e] ``An Overview of Security Support in Named Data Networking" by Spyridon Mastorakis, Zhiyi Zhang, Yingdi Yu, Haitao Zhang, Eric Newberry, Yanbiao Li, Alexander Afanasyev \& Lixia Zhang. NDN Technical Report, NDN-0057, Rev. 2. Apr. 2018.


\bibitem{046}
[Spyridon18f] ``An Overview of Security Support in Named Data Networking" by Spyridon Mastorakis, Zhiyi Zhang, Yingdi Yu, Haitao Zhang, Eric Newberry, Yanbiao Li, Alexander Afanasyev \& Lixia Zhang. NDN Technical Report, NDN-0057, Rev. 2. Apr. 2018.


\bibitem{047}
[Spyridon18g] ``An Overview of Security Support in Named Data Networking" by Spyridon Mastorakis, Zhiyi Zhang, Yingdi Yu, Haitao Zhang, Eric Newberry, Yanbiao Li, Alexander Afanasyev \& Lixia Zhang. NDN Technical Report, NDN-0057, Rev. 2. Apr. 2018.


\bibitem{048}
[Merkle79b]  ``Protocols for Public-Key Cryptosystems" by Ralph Merkle. In 1980 IEEE Symposium on Security and Privacy. Oakland, CA, USA. Apr 1980.


\bibitem{049}
[memphis.edu] ``What is MiniNDN?" by University of Memphis. \texttt{\url{http://minindn.memphis.edu/}}

\bibitem{050}
[Soustroup] ``The Design and Evolution of C++". pp. 207.

\bibitem{051}
[Kutscher15] ``A Survey of Information-Centric Networking" by Dirk Kutscher,  Börje Ohlman, Claudio Imbrenda, Christian Dannewitz \& Bengt Ahlgren. In IEEE Communications Magazine Vol. 50, No. 7. July 2012.

\bibitem{052}
[Afanasyev13c]``Let's ChronoSync: Decentralized Dataset State
Synchronization in Named Data Networking" by Alexander Afanasyev \& Zhenkai Zhu.Proceedings of \nth{21} IEEE International Conference on Network Protocols(ICNP 2013).Göttingen, Germany. Oct 2013.

\bibitem{053}
[Wang18g] ``A Secure Link State Routing Protocol for NDN" by Lan Wang, Vince Lehman, A.K.M. Mahmudul Hoque, Beichuan Zhang, Yingdi Yu \& Lixia Zhang. In IEEE Access, Vol. 6. Jan 2018.


\bibitem{054}
[Wang18h] ``A Secure Link State Routing Protocol for NDN" by Lan Wang, Vince Lehman, A.K.M. Mahmudul Hoque, Beichuan Zhang, Yingdi Yu \& Lixia Zhang. In IEEE Access, Vol. 6. Jan 2018.
 

\bibitem{055}
[Wang18i] ``A Secure Link State Routing Protocol for NDN" by Lan Wang, Vince Lehman, A.K.M. Mahmudul Hoque, Beichuan Zhang, Yingdi Yu \& Lixia Zhang. In IEEE Access, Vol. 6. Jan 2018.
 

\bibitem{056}
[Wang18j] ``A Secure Link State Routing Protocol for NDN" by Lan Wang, Vince Lehman, A.K.M. Mahmudul Hoque, Beichuan Zhang, Yingdi Yu \& Lixia Zhang. In IEEE Access, Vol. 6. Jan 2018.

\bibitem{057}
[Papadimitriou09] ``Real Time Video Streaming over Heterogeneous Networks" by Panagiotis Papadimitriou \& Vassilis Tsaussidis. Proceedings of \nth{11} International Conference On Advanced Communication Technology, 2009(ICACT 2009).Gangwon-Do, South Korea. Feb 2009.


\end{thebibliography}                             %% to generate your bibliography.


\end{document}                                    %% END THE DOCUMENT
